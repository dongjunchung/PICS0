% -*- mode: noweb; noweb-default-code-mode: R-mode; -*-
\documentclass[11pt]{article}
%% Set my margins
\setlength{\oddsidemargin}{0.0truein}
\setlength{\evensidemargin}{0.0truein}
\setlength{\textwidth}{6.5truein}
\setlength{\topmargin}{0.0truein}
\setlength{\textheight}{9.0truein}
\setlength{\headsep}{0.0truein}
\setlength{\headheight}{0.0truein}
\setlength{\topskip}{0pt}
%% End of margins

%%\pagestyle{myheadings}
%%\markboth{$Date$\hfil$Revision$}{\thepage}

\usepackage[pdftex,
bookmarks,
bookmarksopen,
pdfauthor={Zequn Sun, Wei Wei and Dongjun Chung},
pdftitle={PICS Vignette}]
{hyperref}

% \usepackage{fullpage}
% \usepackage{pdflscape}

\title{An Introduction to the `\texttt{PICS}' Package, Version 1.0}
\author{ Zequn Sun,Wei Wei and Dongjun Chung\\
Department of Public Health Sciences, Medical University of South Carolina (MUSC),\\
  Charleston, SC, 29425.}


\date{\today}



\usepackage{Sweave}
\begin{document}
\Sconcordance{concordance:PICS-example.tex:C:/Users/dchung/Google Drive/USB/research - 20. cancer subtype identification/1. pathway_index/software/PICS/vignettes/PICS-example.Rnw:%
1 36 1 1 0 23 1 1 4 1 2 4 0 1 2 19 1 1 2 1 0 1 1 10 0 1 1 5 0 1 1 5 0 1 %
1 12 0 1 2 4 1 1 2 4 0 1 2 15 1 1 2 4 0 1 2 16 1 1 2 1 0 1 1 21 0 1 2 %
15 1 1 2 4 0 1 2 3 1 1 -5 1 9 8 1 1 2 1 0 1 1 18 0 1 2 7 1 1 2 26 0 2 2 %
4 0 1 2 5 1 1 -7 1 11 10 1 1 2 4 0 1 2 3 1 1 -5 1 9 16 1}

%\VignetteIndexEntry{PICS}
%\VignetteKeywords{PICS}
%\VignettePackage{PICS}
%\VignetteEncoding{UTF-8}
\maketitle

\section{Overview}

This vignette provides basic information about the
\texttt{PICS}`\cite{PICS}' package. PICS stands for ``Pathway-guided Identification of Cancer Subtypes''.
The \texttt{PICS} methodology for Pathway-Guided Identification of Cancer Subtypesis developed in \texttt{PICS}. The proposed approach improves identification of molecularly-defined subgroups of cancer patients by utilizing information from pathway databases in the following four aspects.

(1) integration of genomic data at the pathway-level improves robustness and stability in identification of cancer subgroups and driver molecular features; and

(2) summarizing multiple genes and genomic platforms at the pathway-level can potentially improve statistical power to identify important driver pathways because moderate signals in multiple genes can be aggregated; and

(3) in \texttt{PICS}, we consider this ``operation'' or ``interaction'' between pathways, instead assuming that each pathway operates independently during the cancer progression, which may be unrealistic; and

(4) \texttt{PICS} allows simultaneous inference in multiple biological layers (pathway clusters, pathways, and genes) within a statistically rigorous and unified framework without any additional laborious downstream analysis.

The package can be loaded with the command:


\begin{Schunk}
\begin{Sinput}
R> library("PICS")
\end{Sinput}
\end{Schunk}

\section{Input Data}

The package requires that the response consist of 4 components:
(1) z-scores in the form of a either dataframe or matrix; and
(2) time and status variable for Cox model of survival analysis in the form of vectors; and
(3) pathway information is in the form of a list where gene names are included in pathway lists respectively.

The Cancer Genome Atlas (TCGA) is an example application for the `\texttt{PICS}' package.
The TCGA data was downloaded from the cBio Portal (http://www.cbioportal.org/) using the R package `\texttt{cgdsr}', and we used z scores for the mRNA expression data.

All the mRNA expression measures were centered and scaled to have unit variance according to standard practice. Only genes annotated in the KEGG database and that appeared in both datasets were considered for analysis.In the current stage, we focused only on gene expression measurements and aim at incorporating other data types into our model in the future.

The `\texttt{TCGA}' is a list object with four elements, including the `\texttt{geneexpr}' dataframe of z scores for the mRNA expression, the `\texttt{t}' vector of the survival time, the `\texttt{d}' vector of the survival status indicator, and the `\texttt{pathList}' list of the pathway information. The `\texttt{pathList}' has four elements, each of which contains names of genes belonging to each pathway. Z scores for the mRNA expression data of 389 genes are provided for 50 cancer patients, along with survival times and survival statuses.

This dataset can be loaded as follows:

\begin{Schunk}
\begin{Sinput}
R> data(TCGA)
R> TCGA$geneexpr[1:5,1:5]
\end{Sinput}
\begin{Soutput}
        ACLY        ACO1       ACO2         CS       DLAT
1 -2.2410125 -0.48445531 -1.6346455  0.1378804 -3.5310321
2 -2.1301362  0.82116427 -0.9533701  0.6213512  0.6689948
3 -2.9122727 -0.08790649 -1.0975096 -0.2454025 -0.9433900
4 -1.1721514 -0.24249825 -0.7212639  0.1842386 -0.6188785
5  0.5383438  0.98012739 -0.7396043 -0.0699680  1.9573767
\end{Soutput}
\begin{Sinput}
R> TCGA$t[1:5]
\end{Sinput}
\begin{Soutput}
[1] 43.89 40.97 49.12  2.00 46.59
\end{Soutput}
\begin{Sinput}
R> TCGA$d[1:5]
\end{Sinput}
\begin{Soutput}
[1] 1 1 0 1 0
\end{Soutput}
\begin{Sinput}
R> TCGA$pathList[1]
\end{Sinput}
\begin{Soutput}
$KEGG_CITRATE_CYCLE_TCA_CYCLE
 [1] "IDH3B"     "DLST"      "PCK2"      "CS"        "PDHB"      "PCK1"     
 [7] "PDHA1"     "LOC642502" "PDHA2"     "LOC283398" "FH"        "SDHD"     
[13] "OGDH"      "SDHB"      "IDH3A"     "SDHC"      "IDH2"      "IDH1"     
[19] "ACO1"      "ACLY"      "MDH2"      "DLD"       "MDH1"      "DLAT"     
[25] "OGDHL"     "PC"        "SDHA"      "SUCLG1"    "SUCLA2"    "SUCLG2"   
[31] "IDH3G"     "ACO2"     
\end{Soutput}
\end{Schunk}

\section{Pre-filtering}

We expected that only a parsimonious set of genes would be related to patient survival, i.e., the sparsity assumption. To eliminate the most unlikely predictors, we first conducted a supervised pre-filtering by fitting a Cox regression model of each mRNA expression measure on patient survival in the TCGA dataset. Only the gene expressions associated with patient survival at p-values smaller than a pre-specified cut-off were included in the subsequent analysis. Here we chose p = 0.5 as the default cut-off point.

\begin{Schunk}
\begin{Sinput}
R> train.list=prefilter(data=TCGA$geneexpr, time=TCGA$t, status=TCGA$d, plist=TCGA$pathList)